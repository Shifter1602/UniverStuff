\chapter{\textbf{Введение}}
Теория вероятности - математическая наука, позволяющая по вероятностям одних случайных событий находить вероятности других случайных событий, связанных каким-либо образом с первыми. Утверждение о том, что какое-либо событие наступает с вероятностью, равной, например, ½, ещё не представляет само по себе окончательной ценности, так как мы стремимся к достоверному знанию. Окончательную познавательную ценность имеют те результаты теории вероятностей, которые позволяют утверждать, что вероятность наступления какого-либо события А весьма близка к единице или (что то же самое) вероятность не наступления события А весьма мала. В соответствии с принципом "пренебрежения достаточно малыми вероятностями"  такое событие справедливо считают практически достоверным. Поэтому можно также сказать, что теория вероятностей есть математическая наука, выясняющая закономерности, которые возникают при взаимодействии большого числа случайных факторов.\\
Возможность применения методов теории вероятностей к изучению статистических закономерностей, относящихся к весьма далёким друг от друга областям науки, основана на том, что вероятности событий всегда удовлетворяют некоторым простым соотношениям, о которых будет сказано ниже. Изучение свойств вероятностей событий на основе этих простых соотношений и составляет предмет теории вероятностей. Наиболее просто определяются основные понятия теории вероятностей как математической дисциплины в рамках так называемой элементарной теории вероятностей.\\
\chapter{\textbf{Цель курсовой работы}}
Целью данной курсовой работы является углубление теоретических знаний по курсу «Теория вероятностей и математическая статистика», получение навыков самостоятельного решения прикладных задач как на основе теоретических знаний, так и экспериментально с помощью среды R.\\
Курсовая работа содержит решение задач по основным теоремам теории вероятностей и  решение задач по разделу законы распределения случайных величин, оценка их числовых характеристик.
