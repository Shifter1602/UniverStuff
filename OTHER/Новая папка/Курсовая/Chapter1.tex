\chapter{Основная часть}
\section{Часть 1}
\subsection{Условия задач}
Задача 16\\

Сейфовый замок имеет 4 диска с пятью секторами, на
каждом из которых записана одна из цифр от 0 до 4 . Какова
вероятность открыть замок сейфа, набрав 4 цифры наугад?\\


Задача 46\\

Вероятность попадания в кольцо первого игрока – 0,7, а
второго игрока – 0,8. Игроки бросают мяч по два раза независимо друг от друга. Какова вероятность того, что мяч попадет в
кольцо два раза?\\


Задача 58\\

Имеются две коробки с мячами для тенниса. В первой коробке 7 красных и 8 зелёных мячей; во второй – 9 красных и 11
зелёных. Из первой и второй коробок, не глядя, берут по одному
мячу и кладут в третью коробку. Мячи в третьей коробке перемешивают и берут наугад один мяч. Определить вероятность
того, что этот мяч зелёный.\\


Задача 75\\

Для посева заготовлены семена 4 видов клёна. Причем,
22 % всех семян клёна 1-го вида; 33 % – 2-го вида; 32 % – 3-
го вида; 13 % – 4-го вида. Вероятность всхожести для семян
первого вида равна 0,69; для второго – 0,74; для третьего – 0,43;
для четвёртого – 0,38. Найти вероятность того, что наугад взятое
семечко взойдёт.\\


Задача 78\\

В группе спортсменов 12 метателей снарядов, 17 бегунов и
19 прыгунов. Вероятность выполнить квалификационную норму
для метателя снаряда равна 0,71; для бегуна – 0,89; для прыгуна – 0,73. Найти вероятность того, что спортсмен, выбранный
наугад, выполнит норму.\\


Задача 88\\

В лесхозе 50 \% посадок составляет сосна; 40 \% береза и 10 43
\% ель. Вероятность поражения грибковыми заболеваниями для
этих деревьев составляет 0,3; 0,6 и 0,8 соответственно. При санитарном осмотре было выбраковано дерево. Какова вероятность
того, что это дерево ель?\\
