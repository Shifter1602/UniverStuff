\subsection{Теоретическое решение}
Задача 16.\\
Задача решается с помощью классического определения вероятности.\\
Формула классической вероятности имеет следующий вид:\\
\begin{center}
$P=\frac{m}{n}$,\\
\end{center}
где $m$-число случаев, благоприятных для данного
события, а $n$-полная группа равновозможных и несовместных случаев.\\
В сейфовом замке диски независимы друг от друга. Каждый диск имеет по 5 секторов. Следовательно имеем $m=1$ и $n=5$ для каждого из дисков.\\
Пусть $A$ - событие открытия сейфа. Тогда для нахождения $P(A)$ требуется одновременное наступление событий, соответствующих угадыванию цифры на каждом диске:\\
\begin{center}
$P(A)=\frac{1}{5}\cdot\frac{1}{5}\cdot\frac{1}{5}\cdot\frac{1}{5}=\frac{1}{625}=0.0016$\\
\end{center}
Задача 46.\\
В данной задаче применяются теоремы сложения и умножения вероятностей.\\
Пусть $A$ и $B$ - события попадания в кольцо соответственно первого и второго игроков. Тогда вероятности их попадания:
\begin{center}
$P(A)=0.7$\\
$P(B)=0.8$\\
\end{center}
Для подсчета вероятности попадания мяча в кольцо 2 раза из 4 нужно сложить вероятности каждого такого случая.\\
Перечислим данные случаи:
\begin{enumerate}
\item Случай когда первый игрок попал 2 раза, а второй - ни разу. \\$P(A)\cdot P(A)\cdot P(\overline{B})\cdot P(\overline{B})$\\
\item Случай когда второй игрок попал 2 раза, а первый - ни разу. \\$ P(\overline{A})\cdot P(\overline{A})\cdot P(A)\cdot P(A)$\\
\item Случаи когда и первый и второй игроки попали по одному разу(Всего 4 таких случая из-за возможности разных комбинаций.\\
$ P(\overline{A})\cdot P(A)\cdot P(B)\cdot P(\overline{B})$\\
$ P(A)\cdot P(\overline{A})\cdot P(B)\cdot P(\overline{B})$\\
$ P(\overline{A})\cdot P(A)\cdot P(\overline{B})\cdot P(B)$\\
$ P(A)\cdot P(\overline{A})\cdot P(\overline{B})\cdot P(B)$\\
\end{enumerate}
Проведем окончательные численные расчеты:\\
$P=0.7\cdot 0.7\cdot 0.2\cdot 0.2+0.3\cdot 0.3\cdot 0.8\cdot 0.8+4\cdot(0.7\cdot 0.3\cdot 0.8\cdot 0.2)=0.0196+0.0576+0.1344=0.2256$\\

Задача 59.\\
Темой задачи ялвяется полная вероятность.\\
Формула полной вероятности выглядит так:\\
$$P(A)=\sum_{i=0}^{n} P(H_i)P(A/H_i)$$\\
Перекладывание из первой коробки зеленого мяча (событие $H_1$ - первая гипотеза), красного мяча (событие $H_2$ - вторая гипотеза) и соответственно зеленого (событие $H_3$ - третья гипотеза) и красного (событие $H_4$ - четвертая гипотеза) из второй коробки составляют две полные группы независимых событий.\\
Их вероятности равны:\\
\begin{center}
$P(H_1)=\frac{8}{15}$\\
$P(H_2)=\frac{7}{15}$\\
$P(H_3)=\frac{11}{20}$\\
$P(H_4)=\frac{9}{20}$\\
\end{center}
Чтобы посчитать вероятность вытаскивания из третьей коробки зеленого мяча, рассмотрим 3 случая:\\
\begin{enumerate}
\item Из первой коробки достали зеленый мяч, а из второй - красный.
\item Из первой коробки достали красный мяч, а из второй - зеленыый.
\item Из обеих коробок достали по зеленому мячу.
\end{enumerate}
Пусть $A$- событие вытаскивания из третьей коробки зеленого мяча.\\
Найдем условные вероятности для каждого из вышестоящих случаев:\\
В первом и втором случаях в третьей корзине окажутся один зеленый и один красный мячи:\\
\begin{center}
$P(A/H_1 H_4)=P(A/H_2 H_3)=\frac{1}{2}$\\
\end{center}
В третьем случае в коробке будут находиться только зеленые мячи:\\
\begin{center}
$P(A/H_1 H_3)=1$\\
\end{center}
Выполним оставшиеся расчеты по формуле полной вероятности:\\
\begin{center}
$P(A)=\frac{8}{15}\cdot \frac{9}{20}\cdot \frac{1}{2} + \frac{7}{15}\cdot \frac{11}{20}\cdot \frac{1}{2} + \frac{8}{15}\cdot \frac{11}{20}=0.12 + 0.128(3) + 0.29(3)= 0.541 $\\
\end{center}
Задача 75\\
Задача решается методом, аналогичным предыдущей задачи.\\
Пусть $A$ - событие всхода семечка.\\
Пусть $H_1$,$H_2$,$H_3$,$H_4$ - гипотезы, соответствующие выпадению семечка клена каждого из 4 видов.\\
Распишем вероятности  гипотез каждого вида:\\

\begin{center}
$P(H_1)=0.22$ \\
$P(H_2)=0.33$ \\
$P(H_3)=0.32$ \\
$P(H_4)=0.13$ \\
\end{center}
Также запишем условные вероятности всхода каждого вида семян:\\
\begin{center}
$P(A/H_1)=0.69$\\
$P(A/H_2)=0.74$\\
$P(A/H_3)=0.43$\\
$P(A/H_4)=0.38$\\
\end{center}
Для нахождения вероятности того, что семечко, взятое наугад, взойдет воспользуемся формулой полной вероятности:\\
$P=0.22\cdot 0.69 + 0.33\cdot 0.74 + 0.32 \cdot 0.33 + 0.13\cdot 0.38=0.583$\\

Задача 78\\
Задача решается аналогично двум вышестоящим.\\
Пусть $A$ - событие выполнения нормы спортсменом.\\
Пусть $H_1$,$H_2$,$H_3$ - гипотезы, соответствующие выпадению каждого типа спортсменов.\\
Вероятности гипотез каждого вида:\\
\begin{center}
$P(H_1)=\frac{12}{48}=\frac{1}{4}$ \\
$P(H_2)=\frac{17}{48}$ \\
$P(H_3)=\frac{19}{48}$ \\
\end{center}
Условные вероятности выполнения нормы спортсменами:\\
\begin{center}
$P(A/H_1)=0.71$\\
$P(A/H_2)=0.89$\\
$P(A/H_3)=0.73$\\
\end{center}

Для нахождения вероятности того, что спортсмен, взятый наугад, выполнит норму воспользуемся формулой полной вероятности:\\
$P=\frac{1}{4}\cdot 0.71 + \frac{17}{48}\cdot 0.89 + \frac{19}{48}\cdot 0.73=0.7816$\\
\\Задача 88.\\
В задаче используется формула Байеса (теорема гипотез).\\
Пусть имеется полная группа несовместных гипотез $H_1,H_2,\cdots,H_n$, известны вероятности $P(H_i)$,  $i=\overline{1,n}$ этих гипотез до опыта и условные вероятности
$P(A/H_i)$  события $A$, которое может произойти лишь совместно с одной гипотезой.
Произведён опыт, в результате которого событие $A$ произошло, тогда условные вероятности $P(H_i/A)$ вычисляются по формуле:\\
\begin{center}
$P(H_i/A)=\frac{P(H_i)P(A/H_i)}{P(A)}$,
где $$P(A)=\sum_{i=0}^{n} P(H_i)P(A/H_i)$$\\
\end{center}
Пусть $A$ - событие выбраковки дерева.\\
Пусть гипотезы $H_1$,$H_2$,$H_3$ соответствуют выпадению каждого вида дерева.\\
Имеем следующие вероятности данных гипотез:\\
\begin{center}
$P(H_1)=0.5$\\
$P(H_2)=0.4$\\
$P(H_3)=0.1$\\
\end{center}
Условные вероятности выбраковки каждого вида дерева:\\
\begin{center}
$P(A/H_1)=0.3$\\
$P(A/H_2)=0.6$\\
$P(A/H_3)=0.8$\\
\end{center}
Вычисляем $P(A)$ и подставляем в формулу Баейса:\\
\begin{center}
$P(A)=0.5\cdot 0.3+0.4\cdot 0.6+0.1\cdot 0.8=0.47$\\
$P(H_3/A)=\frac{P(H_3)P(A/H_3)}{P(A)}=\frac{0.1\cdot 0.8}{0.47}=0.1702$\\
\end{center}
\newpage

