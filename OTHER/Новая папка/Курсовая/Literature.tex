\chapter{Список литературы}
\begin{enumerate}
\item Гмурман В.Е. Руководство к решению задач по теории вероятностей и математической статистике: учеб.пособие -
М.: ИД Юрайт, 2010. - 404 с.\\
\item  Данилин Г. А. и др. Элементы теории вероятностей с Excel:
Практикум для студентов всех специальностей МГУЛа.,/
Г. А. Данилин, В. М. Курзина, П. А. Курзин, О. М. Полещук. М.: МГУЛ, 2004. 87 с.: ил.\\
\item  Вентцель Е. С. Задачи и упражнения по теории вероятностей: Учеб. пособие для студ. втузов / Е. С. Вентцель, Л.\\
А. Овчаров. — 5-е изд., испр. — М.: Издательский центр
«Академия», 2003. — 448 с.\\
\item  Венцель Е.С. Теория вероятностей: учебник - М.: Физ-мат
лит., 1958. - 464 с.\\
\item Гнеденко Б. В. и Хинчин А. Я., Элементарное введение в теорию вероятностей, 3 изд.,К. - Л.,2008.\\
\item Луговая И. Н., Курс теории вероятностей, 4 изд., М., 2001.\\
\item Феллер В., Введение в теорию вероятностей и её приложение (Дискретные распределения), пер. с англ., 2 изд., т. 1-2,К.,2003.\\
\item Бернштейн С. Н., Теория вероятностей, 4 изд.,К. - Л., 2003.\\
\item А.А.Савельев и др., Основные понятия языка R Учебно-методическое пособие, Казань 2007
\end{enumerate}